\chapter{Codification}

\section{Portability}

The code of this library is supposed to be portable for a large variety of C++ compilers and
machines. All all developers to keep it in mind during their work.

The following rules must always be followed:

\begin{itemize}
	\item Do not rely on your knowledge about data type sizes because they may vary from
		system to system and even from compiler to compiler ({\bf int} type is a
		good example, it may vary fom 16 to 64 bits). Always use the {\bf sizeof} operator.
	\item Do not make any assumptions about numeric formats. You never know how the target
		system will store them in memory.
	\item Avoid assembly blocks or better, do not use them. It will not be as worthy as it may
		appear. The modern compilers can produce very optmized codes. If you do it, make sure
		you will protect it with an ``\#ifdef'' macro and to create an equivalent C++ source
		code for other platforms.
	\item Avoid use of system specific functions. Use the standard C/C++ functions if available
		(ANSI C/C++). Don't forget to protected the system specific code with an ``\#ifdef'' and
		to write a portable version of the algorithm based on the standard functions.
\end{itemize}

Some standard C/C++ library headers may contain different sets of functions, because of that
include blocks may be protected by a ``\#ifdef'' macro for each supported compiler.

\section{Conditional Compilation}

Some features of this library may degrade the performance of the implemented MAMs or may
lead to undesirable behavior in user applications. Because of these, they must be
implemented as optional features during the compilation time.

In the current version, there are 3 features which must be considered optional:

\begin{itemize}
	\item Extra Debug information: This feature enables the debug features hidden inside the code.
		They may provide extra checking and verbose messages. They compromise the performance of
		the library. The code pieces releated with this feature are protected by the macro
		{\bf \_\_stDEBUG\_\_}.
	\item Performance statistics: This feature enables the support for statistics. Some of them
		are always available because they are very cheap but other will not be compiled when this
		feature is not enabled. The code pieces releated with this feature are protected by the
		macro {\bf \_\_stSTATISTICS\_\_} (not implemented yet).
	\item Visualization Module: The MAMView visualization support. This feature enables the
		visualization support in some structures. This feature is very expensive and should not be
		enabled when it is not required. The code pieces releated with this feature are protected
		by the macro {\bf \_\_stVIEW\_\_} (under development).
\end{itemize}

The support for each feature is controlled by the definition of certain macros and their
code must be protected by a block of ``\#ifdef'' preprocessor macros.

\section{Code Style}

The code is formatted using 3 spaces for each level of indentation. All closing brackets,
except the one before an else, must have an ending block comment.

\section{Module Separation}


\section{Documentation}

This library uses the Doxygen documentation tool. The comment blocks must be always in headers
except for items


